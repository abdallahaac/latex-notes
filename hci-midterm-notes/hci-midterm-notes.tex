\documentclass{article}
\begin{document}

\title{Midterm Review}
\author{HCI-IMD3004}
\maketitle




\section*{Overview}
The Exam will have the following structure:
\begin{enumerate}
\item Multiple choice (15 question) 1 points
\item Short answer (4 question ) 2 points
\item Long (2 question) 5 points
\end{enumerate}
Checklist
\begin{enumerate}
\item Anova T-test (Excel)
\begin{enumerate}
    \item check T-Test from tutorial 
    \item Check p-value
    \item Recreate it 
\end{enumerate}
    
\item Youtube Videos
\begin{enumerate}
    \item Watch Videos
    \item Extract Transcripts from video
    \item Summarize key points in pdf
\end{enumerate}
\item Draw or Miro Board
\begin{enumerate}
    \item Could be storyboarding?
    \item Could be creating user interface?
    \item Open Miro Board
    \item Open Clip Studio Paint 
    \item Whiteboard ready
\end{enumerate}

\item Reread Pdf document

\end{enumerate}

\newpage
\section*{Topics}
What IS on the Exam:
\begin{itemize}
\item Interactivity 
\begin{itemize}
    \item Reactive, interactive, conversational
\end{itemize}

\end{itemize}
\begin{itemize}
\item Biases/Assumptions


\end{itemize}
\begin{itemize}
\item Political Artefacts 

\end{itemize}


\begin{itemize}
\item Paradigms 
\begin{itemize}
    \item Paradigm 1
    \item Paradigm 2
    \item Paradigm 3
\end{itemize}
\end{itemize}
\begin{itemize}
\item System, Mental, Manifest Models 
\end{itemize}
\begin{itemize}
\item 6 Principles of Design
\begin{enumerate}
    \item Visibility 
    \item feedback 
    \item affordance
    \item mapping, 
    \item constraints
    \item  consistency
    \end{enumerate}
\end{itemize}

\begin{itemize}
    \item Hierarchical Task Analysis 
\end{itemize}
\begin{itemize}
    \item T-test and ANOVA
\end{itemize}


\newpage

\section{Interactivity}

Interactivity can be classified into three types: reactive, interactive, and conversational. Reactive interactivity has a fixed response to input, while interactive interactivity responds dynamically to input, and conversational interactivity involves a series of interactions with system agency.

\subsection{Reactive}
Reactive interactivity refers to a type of interaction where input is directly linked to a fixed output response, without any opportunity for user intervention or modification. The output response is predetermined and unchanging, and there is no variation based on contextual factors such as time, location, or input force.
\subsection{Interactive}
Interactive interactivity is a type of interaction in which the input from the user affects the output, but the response is dynamic and can change based on a range of parameters, such as time, location, or force applied. In this type of interactivity, the user has a degree of control over the system's response.
\subsection{conversational}
Conversational interactivity refers to an interaction in which the input from the user influences the system's output, and the system has agency to generate responses based on a series of interactions. In other words, the system is capable of engaging in a conversation with the user, where the responses are not fixed and can vary based on the context and previous interactions.


\section{Biases/Assumptions}

\subsection{Biases}
Unconscious bias refers to an association or thought that we are unaware we have, reflecting on who we are and our perspectives, and questioning how those perspectives are formed.
\subsection{Assumptions}
Assumptions are beliefs or perspectives that we may hold unconsciously or without questioning, which can lead to biases in our decision-making and design processes. It emphasizes the importance of reflecting on our assumptions and questioning how they are formed, as well as declaring them upfront and testing their validity. The text also suggests that acknowledging and addressing assumptions can reduce project risk early in the iteration process.

\section{Political Artefacts}
The idea that artifacts, also known as technology, can have political aspects and are not neutral. The author argues that the design of technology is influenced by political ideologies and biases. The distribution of power, authority, and privilege in a community is what the author refers to as politics. The paper discusses how decision-making during the design of artifacts can make them political, but the artifacts themselves are not political. The author explains that artifacts have politics by necessity, meaning they require specific needs to work. The text also discusses the ethical considerations that should be taken into account when designing technology interventions, such as considering the impact on different levels of society and designing for inclusivity. Finally, the text emphasizes the importance of acknowledging the existence of political artifacts and the need for better designs.


\section{Paradigms}

\subsection{Paradigm 1: Human Factors}
The human factors paradigm focuses on identifying problems in man-machine coupling and aims to optimize the fit between humans and machines. This paradigm originated from ergonomics and industrial systems research, and it prioritizes identifying and solving concrete problems that cause disruption in interaction.

\subsection{Paradigm 2: Classical Cognitivism/Information Processing }
This paradigm views human information processing as analogous to computational signal processing and aims to enable communication between the machine and the person. It models the state of the person and the computer to predict and optimize their relationship. This paradigm dominates the HCI field and is characterized by a set of information processing phenomena or issues in computers and users.
\subsection{Paradigm 3: Phenomenologically-Situated}
This paradigm involves exploring phenomena and issues that are marginalized by the first two paradigms. It grapples with alternative perspectives that arise from dynamic use contexts, socially situated interactions, and indirect and multiple goals. The goal of this paradigm is to describe identified factors and conditions and investigate the meaning of interaction in complex settings.


\section{System, Mental, Manifest Models}
\subsection{System Mode}
The system mode is how a system or machine truly works, as designed and built by engineers. It is concerned with the technical implementation of the system, such as the underlying hardware and software. The system mode can be seen as the "reality" of the system, as it exists independently of the user's understanding or experience. For example, in the case of a pin tumbler lock, the system mode describes the mechanics of how the lock and key interact to allow or prevent access. Understanding the system mode is important for engineers and technical experts who need to design, build, and maintain the system.
\subsection{Mental Mode}
The mental model is how a user thinks the system or machine works based on their own understanding and experience. It is the user's perception or belief about how the system works, which may or may not be accurate. Mental models can differ from the system mode and may be influenced by factors such as prior knowledge, expectations, and user interface design. For example, a user's mental model of a pin tumbler lock may be that they need to match the shape of the key to the lock to open it. Understanding the user's mental model is important for designers and developers who need to create user interfaces and interactions that are intuitive and easy to use.
\subsection{Manifest/Interface Models}
The interface mode is how the system or machine presents itself to the user through its user interface or design. It is the designer's intended conveyance of the system mode to the user, and may or may not align with the user's mental model. Interface modes can vary depending on the system or machine, but often involve visual and interactive elements that allow the user to interact with the system. For example, the interface mode of a pin tumbler lock might be a physical key and lock mechanism that users manipulate to open and close the lock. Understanding the interface mode is important for designers and developers who need to create user interfaces that effectively convey the system mode and align with the user's mental model.

\section{Principles of Design}

\subsection{Visibility}
The visibility principle of design emphasizes that the more visible an element is, the more likely users will know about them and how to use them. It suggests that designers should make all elements of the system visible to the user so that they can easily find and use them. For instance, having a bright color for an essential button makes it more visible, which helps users locate it quickly. This principle is critical in creating a user-friendly interface, as it ensures that users can navigate the system easily and efficiently.
\subsection{Feedback}
The feedback principle of design is about making it clear to the user that action has been taken and what has been accomplished. Feedback assures users that their interaction with the system is producing the desired results, giving them the confidence to continue using it. Designers can achieve this principle by providing audio or visual feedback, such as a beep or a clipping sound, or by showing the results of the action in the system. Feedback is crucial in building trust between users and the system, improving user satisfaction and experience.
\subsection{Affordance}
The affordance principle of design refers to an attribute of an object that allows people to know how to use it. It suggests that users should be able to understand how to use the system by merely looking at it, based on the system's design. For instance, a button with a curved shape suggests that it can be pushed, and a rectangular shape suggests it can be slid. This principle helps users interact with the system more intuitively and efficiently, improving the system's overall usability.
\subsection{Mapping}
The mapping principle of design involves having a clear relationship between controls and the effect they have on the world. It suggests that the controls should be placed in a way that makes it clear what they affect, and the effect should be predictable. For instance, in a video game, the left joystick is typically mapped to movement, while the right joystick is mapped to camera control. This principle helps users understand the system more quickly, as they can predict the effect of their actions, and also helps avoid confusion and errors.
\subsection{Constraints}
The constraints principle of design is about limiting the range of interaction possibilities for the user to simplify the interface and guide the user to the appropriate next action. Constraints make it easier for the user to navigate the system, reducing the complexity of interactions. For instance, a drop-down menu constrains users to a set of options, helping them select the appropriate choice quickly. This principle helps users focus on the essential elements of the system, improving their efficiency and overall experience.
\subsection{Consistency}
The consistency principle of design involves having similar operations and similar elements for achieving similar tasks. It suggests that designers should strive for consistency in the system, creating uniformity in design, behavior, and interaction. For instance, having the same design elements in different pages of the system, such as using the same color or button shape, helps users recognize that they are in the same system. This principle helps users navigate the system quickly, improving their efficiency and reducing confusion.


\section{Hierarchical Task Analysis (HTA)}
Hierarchical Task Analysis (HTA) is a method used in Human-Computer Interaction to describe how users perform complex tasks. HTA is a systematic and structured method that breaks down a complex task into smaller and more manageable sub-tasks or steps. It involves creating a hierarchical structure of goals, sub-goals, and individual tasks. The method aims to capture the cognitive processes and strategies involved in a user's decision-making and problem-solving processes. HTA can be represented in various forms, including flow charts or lists, with each step representing a task in the process.

For example, if the task is to make a cup of tea, the HTA would include goals such as "prepare the water," "prepare the teabag," "prepare the cup," and "pour the water." Sub-tasks under "prepare the water" may include "fill the kettle with water," "turn on the kettle," and "wait for the kettle to boil." Each of these sub-tasks can be further broken down into smaller steps. The hierarchical structure of HTA makes it easier to identify potential errors and inefficiencies in the task and to design better interfaces that improve user experience.

\section{Anova T-Test}



\section{Youtube Videos}
\subsection{Petcube Bites Treat Camera: Designed for Pet Parents}
The video is promoting the "Petcube Bites" camera, which allows pet owners to remotely interact with their pets. The camera has a built-in speaker and microphone, which enables the owner to talk to and hear their pet. The device can dispense treats, which can be flung as a reward for good behavior. The camera also has sound and motion alerts to notify the owner if there is trouble. With multiple cameras, the owner can follow their pet from room to room. The camera can also capture cute moments and share them with friends and family. The device can be scheduled to dispense treats at specific times, letting the pet know when the owner will be around. The video ends with the message that Petcube Bites is designed for happy reunions and for pet parents.
\subsection{Meditation Headpsace}
The video promotes Headspace, a guided meditation app for everyone, regardless of experience level. Meditation has been shown to reduce stress, improve focus, and facilitate healthy sleep. The app offers exercises for different topics, such as sleep or self-esteem, and suggests exercises based on the user's preference. Subscribing gives access to hundreds of guided meditations from the Headspace library, including mini meditations for when you're short on time, exercises to promote mindfulness throughout the day, and packs to help with relationships, sports, sleep, and more. Headspace is promoted as a personal meditation guide for everyone.

\end{document}
