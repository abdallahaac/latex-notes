\documentclass{article}


\title{ \small Notes on Terry Pinkard's \\
\large German Philosophy\\ 
 1760 -1860 \\
The Legacy of Idealism
}

\author{ by Abdallah Abou-Chahine }
\date{\today}

\begin{document}
\maketitle

\newpage
\section*{Note to the Reader}

This document is a collection of my thoughts while studying German Idealism through Terry Pinkard’s book “The Legacy of Idealism.” I have written this document as a means to understand the material and share my insights and opinions with others who may be interested in this subject. 


Please note that these notes are a personal reflection on Terry Pinkard's book "The Legacy of Idealism." They are not intended to be a scholarly source or a substitute for reading the book itself. Rather, they are my own thoughts and interpretations of the material, and should be read as such. 
I encourage readers who are interested in the subject to read Pinkard's book for a more comprehensive and authoritative account of the topic.


\newpage
\part{Introduction:`Germany'' and German Philosophy}

\renewcommand{\thesubsection}{\Roman{subsection}}

 Pinkard starts the introduction with a brief overview of the ``Seven Year War'' contextualizing the German Land of Prussia as a major European power resulting from the war fought on ``German soil''.
  At that time, Germany did not exist as a unified whole but was more or less recognized by the German-speaking parts of the slowly expiring ``Holy Roman Empire of the German Nation''.
  Pinkard therefore proposes that ``Germany'' during that period must be put into quotation marks since there was no such thing as ``Germany''. 


  The only cultural presence that ``Germany'' had back then was its severe religious division of Protestant and Catholic areas, along with the conflicts that emerged from that division. 
  However, by the end of the 18\textsuperscript{th} century, ``German'' philosophy dominated Europe and changed the shape of how the world sees itself, nature, human history, knowledge, politics and the human mind in general. 


  \subsection{The Emergence of Enlightened Thought}

During the middle of the 18\textsuperscript{th} century there was a significant increase in ``Germany's'' population due to commercialized agriculture. 
Because of this growth in population, local princes (the authority of the Holy Roman Empire disspiated and was replaced by local rulers) came to require more money to sustain their courtly life which the French had set as a model. 
The princes tried their best to imitate Versailles' royal court by demanding balls, palaces and so on. These demands were expected to be fulfilled by an efficient bureaucracy trained in the latest management techniques.
Consequently, the princes looked towards the universities. 

These expectations from the universities were slowly progressing the cultural state of ``Germany'' and introduced Enlightened thought. With the steadily increase of the population, the universities became filled with young men 
trained in the ideas of the Enlightenment. However, the economic state of ``Germany'' could not afford to offer sufficent employment opportunities for these men. As a result,``reading clubs'' became a frequent occurrence for those who hoped to join the ranks of
the bureaucratic administration. These reading clubs became so popular that for those who attended them frequently were considered to be diagnosed with a new illness - ``reading addiction''.
The reading clubs gave birth to a new culture that was outside of the ``bourgeois'' and ``popular'' life. For the most part, the emerging culture was considered to be a mode of cultivation and self-directed learning - \textit{Bildung}.

Pinkard says ``to acquire \textit{Bildung} was also to be more than educated; one might become `educated', as it were, passively, by learning things by wrote or by acquiring the abilityto mimic the accepted  opinions of the time.''

\subsection{The Passions of Young Werther}
At 23, the young brilliant Johann Wolfgang Goethe took Europe by storm with his publication of \textit{The Passions of Young Werther}. A novel about a young man named Werther who deeply falls in love with a young woman named Charlotte (Lotte). 
The story ends tragically with Werther ending his life following the unreciprocated feelings from Lotte. 
The book made the German people aware of their intimate dialectical experience of alienation. On one hand, they experienced a subjective sensibility that they believed would free them from alienating social circumstances. On the other hand, this very sensibility became alienating and dangerous, as they realized the limitations and conflicts it presented.
It was inevitable for the book to be as popular as it was - for it exposed the dual consciousness of man and brought the audience into a higher synthesized state.
\newpage
\begin{center}
\section*{\small PART II}
\section*{\textit {Kant and the revolution in philosophy}}

\end{center}
\newpage

\begin{center}
\section*{\small CHAPTER I}
\section*{The revolution in philosophy (I):\\human spontaneity and the natural order}
\vspace{30mm}
\subsection*{\small FREEDOM AND CRITICISM}
\end{center}
One of the most popular aphorisms in philosophy is in Kant's ``An Answer to the Question of Enlightement'' where he equates the Enlightenment with ``man's release from his \textit{self-incurred} immaturity...the inability to use one's own understanding without guidance of another''.\\

\begin{itemize}
\item an increase of \textit{self-incurred} immaturity with the rise of AI technology .i.e man can no longer think for himself. 
\item man has become estranged from his own reason
\item incapacity to think for 
\item death drive - enjoyment
\end{itemize}

Kant emphasizes that this "self-incurred" state is not in fact natural but that we brought it upon ourselves. Kant went on to say that there was only one thing needed for an enlightememnt of this kind - freedom.
All that was required for this to come about was simply to have to the ``courage'' to do so.
\begin{itemize}
\item why  \textit{freedom}?
\begin{itemize}
  \item because it is inherently independent
  \item man will be free to think for himself
\end{itemize}
\item courage to be free?
\begin{itemize}
  \item this presuposses that man is resistant to think for himself - he is resistant towards his own freedom
\end{itemize}
\end{itemize}

\end{document}
